\documentclass{tubaf-thesis}
\usepackage[sans]{tubaf-fonts}
\usepackage{graphicx}
\usepackage{blindtext}
\usepackage[english]{babel}
\usepackage{tabularray}
\UseTblrLibrary{booktabs}
\usepackage[style=numeric, backend=biber, sorting=none]{biblatex}
\addbibresource{bachelorarbeit.bib}

% Document
\begin{document}
    \title{Design and Development of a Mobile Application for Citizen Science-Based Landslide Oberservation and Data Collection}
    \author{Marvin Menzel}
    \date{\today}


    \supervisors{Prof. Dr. S. Zug \and Cornelia Kitzig}
    %

    \examiners{Prof. N. Naseweis \and Dr. S. Schlauer}
    \course{Angewandte Informatik}

    \logo{\includegraphics{innovative-hochschule}} % Logo of Institute, Project etc.

    \maketitle

    \tableofcontents
    
    \chapter{Introduction}
	Landslides pose a significant and growing threat globally, driven by a complex interplay of demographic pressure and rapid land use and land cover changes. 
	These factors collectively contribute to an increased frequency and impact of landslide events.
	The human toll is particularly stark, with a proportional rise in fatalities observed as human stress on slopes intensifies, especially under extreme weather conditions.

	Effective recording, monitoring, and analysis of landslide occurrences are therefore essential for developing robust and predictive models.
	This is particularly crucial in regions like South Asia, which is profoundly affected by landslides, largely due to the intense summer monsoon season.
	Globally, the majority of fatal non-seismic landslides are triggered by natural factors (84.5\%), with anthropogenic causes accounting for a smaller but still significant 15.5\%.
	
	Landslide concentration and associated fatalities are highest in tropical and temperate regions, accounting for 94\% of all recorded events.
	Interestingly, the spatial distribution of landslide triggers varies: anthropogenic landslides show a concentration that is about 20\% higher in plains, whereas natural causes are more prevalent in low to middle mountain ranges.
	While the mortality rate (average number of fatalities per landslide) is higher for naturally triggered landslides, it's lower in tropical and temperate regions.
	It's also worth noting that landslides resulting in numerous fatalities are relatively rare occurrences.

	Looking ahead, climate change is projected to further exacerbate the landslide problem, leading to an expected increase in events.
	While many landslides are naturally triggered, anthropogenic factors, such as reduced hill stability due to human activities, can significantly influence their occurrence and impact.
	Generally, natural landslides tend to occur at higher elevations, while those with anthropogenic causes are more common at lower elevations. \cite{fidan_understanding_2024}
	
    \chapter{Related Work}
    	\section{Data collection}
    	Citizen science has increasingly been recognized as a valuable complement to traditional scientific data collection. A notable example comes from Australia, where adding passively collected citizen science data to scientist-generated datasets significantly improved distribution models of invasive rabbits. This highlights the potential benefits of citizen science approaches, particularly in enhancing spatial coverage and capturing data from areas that would otherwise remain inaccessible to researchers. For instance, private properties, which are often closed to professional scientists, can become valuable sources of ecological data when citizens are actively involved.
    	
    	In mosquito monitoring, citizen science can be especially efficient due to its passive nature and its proximity to human populations, where mosquito encounters are most frequent. However, comparisons with professional approaches reveal clear differences. While passive citizen-based monitoring excels at detecting species near populated areas, active monitoring methods such as trapping remain more effective for capturing species diversity and conducting systematic surveillance. Moreover, passive citizen science approaches are less suitable for short-term surveillance efforts, studies of spatially restricted regions, or investigations of selected habitats that require targeted sampling strategies.
    	
    	Despite these limitations, citizen science has the potential to play an increasingly important role in ecological monitoring and public health surveillance. Its integration should not be seen as a universal solution, but rather as a supplementary method that enriches and extends professional data collection. The combined use of both approaches may therefore provide the most robust framework for monitoring invasive and disease-vector species such as mosquitoes. \cite{pernat_citizen_2021}
    	
    	The success of deep learning systems is heavily dependent on the quality and preparation of training data. It is estimated that 80–90\% of the time spent in machine learning development is devoted to data preparation, underscoring the importance of this stage in the pipeline. Even the most sophisticated algorithms cannot yield reliable results without high-quality datasets. In fact, the accuracy and generalizability of models are strongly constrained by the quality, consistency, and representativeness of the data they are trained on.
    	
    	Consequently, challenges in data collection—such as incomplete coverage, noise, or sampling bias—directly translate into reduced model performance. For domains like mosquito monitoring, where citizen science and professional data are combined, ensuring data reliability becomes a critical task. High-quality data preparation and validation processes are therefore indispensable to fully exploit the potential of deep learning methods in ecological research and public health applications. \cite{whang_data_2020}
    	
    	
    	\section{Citizen Science and Landslides}
    	Citizen science is "scientific work, for example collecting information, that is done by ordinary people without special qualifications, in order to help the work of scientists" \cite{haklay_neogeography_2013,haklay_what_2021}
    	In the context of landslide risk management, local people can act as “human sensors,” providing valuable on-the-ground observations and reports. However, this approach also faces significant challenges, including the low accuracy of volunteered data, language barriers, and inconsistencies in reporting. To be effective, contributions should ideally capture essential details such as date, time, location, and the potential trigger of the event.
    	
    	A comprehensive strategy for disaster risk reduction requires addressing several interconnected requirements. These include conducting thorough landslide risk assessments, integrating findings into land-use planning and zoning regulations, and developing appropriate building codes and standards. In addition, early warning systems, education and awareness programs, as well as community involvement through citizen science initiatives, are vital components for increasing resilience and preparedness.
    	
    	The management of landslides can be structured into three stages: pre-event, during the event, and post-event. In the pre-landslide stage, preventive measures such as planting vegetation, directing surface runoff, or installing flexible pipes can significantly reduce landslide risk. Preparedness activities are also crucial, including the establishment of emergency plans (e.g., in mobile applications), assembling survival kits with essential items such as documents, medicines, clothes, food, and maps, and utilizing digital tools like the Landslide Tracker App or the AmritaKripa App. During a landslide event, priority should be given to evacuating all community members to safe ground or nearby relief camps, with particular attention to vulnerable groups such as pregnant women, children, and the elderly. In the post-event stage, damage assessments play a central role in guiding rehabilitation and recovery processes.
    	
    	Despite these measures, several challenges hinder effective landslide risk management. These include a general unwillingness to relocate, the inefficient transfer or even loss of indigenous knowledge, limited perception of risk among affected populations, and security issues in relief camps. Additionally, resource and expertise constraints, as well as data availability and quality issues, further complicate both scientific and community-driven efforts.
    	
    	Nevertheless, the potential of citizen science in this field continues to grow, particularly with the increasing penetration of internet access and mobile technologies. By harnessing this potential, disaster risk management can become more inclusive, collaborative, and adaptive to local contexts.
    	
    	Do you want me to shape this more like a literature review section (with a neutral summary of studies) or like a discussion/argument section (where you emphasize problems and propose solutions)?
    	
    	
    	
    	\section{Technical aspects of data synchronization}
    	
    	A big part for the success of this project is the synchronization of the collected data. Especially with Nepal being a developing country, a decent network connection is not always given.
    	Therefore it is important to implement a solid way of storing the collected data locally and ensuring a robust and secure service that synchronizes it reliable, consistent and efficient.
    	
    	Common problems of failure are loss of connection, application crashes and loss of power. These problems can lead to the creation of multiple copies of instances, dangling pointers and corrupted files that might even spread to other devices. \cite{noauthor_citizen_2025}
    \chapter{System Specifications}
    	\section{Hardware}
    		\subsection{Sensors}
    		\subsection{Storage}
    		\subsection{Energy}
    	\section{Software}
    		\subsection{Operating Systems}
    		\subsection{User Interface {\&} Experience}
    	\section{Constraints and considerations}
    		\subsection{Network Coverage}
    		\subsection{Localization}
    		\subsection{Timestamps}
    \chapter{Implementation}
    	\section{Core Features}
    	\section{Data upload}
    	
    	\section{Challenges}
    \chapter{Evaluation}
    	\section{Tests}
    		\subsection{Functionality}
    		\subsection{User Experience and Usability}
    \chapter{Discussion}
    	\section{Process}
    	\section{Ethical and Data Privacy aspects}
    \chapter{Conclusion {\&} Outlook}
    	\section{Summary}
    	\section{Future Work}

    \makedeclarationofauthorship

    \nocite{*} % include all entries in the bibliography for testing
    \printbibliography

\end{document}
